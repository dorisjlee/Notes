\documentclass[12 pt , twoside, letterpaper] {article}
\usepackage{geometry}
\newgeometry{margin=1.5cm}
\usepackage{amsmath}
\usepackage{amsfonts}
\usepackage{amssymb}
\usepackage{graphicx}
\usepackage[normalem]{ulem}
\renewcommand{\vec}[1]{\mathbf{#1}}
\let\oldhat\hat
\renewcommand{\hat}[1]{\oldhat{\mathbf{#1}}}
\usepackage{wasysym}     

\begin{document}
\title{Giancoli Ch20: Second Law of Thermodynamics}
\date{}
\maketitle
\vspace{-50pt}
	\section{2nd law of thermodynamics}
		\begin{itemize}
			\item 1st law of thermo $\rightarrow$ directionless energy conservation
			\item 2nd law tells us that spontaneous process is irreversible
			\item for thermodynamical processes,\textbf{heat flows spontaneously from hot $\rightarrow$ cold, not the reverse}
			\item more generally, in any type of processes, we can define 2nd law in terms of entropy 
			%%%%%%%%%%%%later%%%%%%%%%%%%%%%%%%
			
			
		\end{itemize}				
	
	\section{Heat Engines}
		\begin{itemize}
				\item \textbf{Heat engine} : any device that change thermal energy $\rightarrow$ mechanical work
				\item while $Q_H \rightarrow Q_L$, Q$\rightarrow$ W : $Q_H= W+Q_L$
				\item W always positive
				\item consider cyclical system that return to initial every cycle
				\item ex) Steam Engine
					\begin{itemize}
						\item working substance : material that is heated and cooled  (e.g. steam)
						\item simmilar design except work is done through different ways
						\item Reciprocating type  : W done to move piston
						\item Turbine 
					\end{itemize}
				\item Internal Combustion engine: repeated four-stroke cycle of fuel mixture ignition and moving piston
		\end{itemize}			
	




\end{document}