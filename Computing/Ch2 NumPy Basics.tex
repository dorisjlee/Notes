\documentclass[12 pt , twoside, letterpaper] {article}
\usepackage{geometry}
\usepackage{cancel}
\usepackage{amsmath}
\usepackage{amssymb}
\newgeometry{margin=1cm}
\begin {document}

\title{Ch2 : Beginning with NumPy Fundamentals} 
\date {\today}
\maketitle
\footnote{NumPy 1.5 Beginner Guide by Ivan Idris}
\section{ndarray}
\begin{itemize}
\item multi-dimensional array object called ndarray
\item  consists of 1) The actual data
2) Some metadata describing the data
\item many operations change only the metadata 
\item arange is 1D array 
\item $\because$ NumPy array is homogeneous (items must be same type)
\item $\therefore$ easy to determine storage size for array 
\item indexing like Python 
\item ndarry has two attributes:
\item dtype: NumPy datatypes represented by special objects \footnote{ myobject.dtype(): function that returns the datatype of the objects}
\item shape \footnote{myobject.shape(): returns the shape of the object}
The shape attribute of the array is a tuple,  contains the length in 
each dimension.
\item the \verb! array! function takes in any object that is array-like (e.g. Python lists) then create an array from the object.\footnote{NumPy function tends to have many various optional args with predefined defaults.}


\end{itemize}
\section {Data Types}
\begin{itemize}
\item NumPy support much more data types than Python library $\therefore$ program can be optimized (least memory) to specific numerical usage, also $\because$ complex numbers
\item There is conversion methods from each type to another \footnote{Exception: TypeError if try to convert ComplexNumber  into an integer, or a float }
\\ ex) \verb! In: float64(42)!
\\ \verb !Out: 42.0!
\item Many functions have a data type argument, which is often optional. \footnote{ex) for array $\because$ only one data type $\therefore$ autodetect type}
\item  Character codes are included for backward compatibility with Numeric. 
\item Their use is not recommended, should instead use dtype objects.
\item dtype constructor: create data types, can take character code or just its general name
\item A listing of all full data type names can be found in sctypeDict.keys()
\item dtype attribute:
\begin{itemize}
\item d.char : return the character code of dtype object
\item  d.type: attribute corresponds to the type of object of the array elements
\item d.str : string representation of the data type. \\ \verb!`<endianness-optional> <character-code><number-of-bytes-each-array-item-requires>'! Endianness = the way bytes are ordered within a 32 or 64-bit word. In big-endian order, the most significant byte is stored first. In little-endian order, the least significant byte is stored first.

\end{itemize}
\item Example: Creating a record data type \footnote{record data type is a heterogeneous data type like a row in a spreadsheet.}
\end{itemize}
\section{Slicing and Indexing ndarray}






\end{document}