\documentclass[12 pt , twoside, letterpaper] {article}
\usepackage{geometry}
\newgeometry{margin=1cm}
%\usepackage{}
%this place is called preamble
\begin {document}
\title{Intro to CSS}
%\author{}
%\date {\today}
%\maketitle
\date {\today}
CSS has a selector and declarations
\section{Syntax}
\verb! h1{color:blue' font-size :12px}!
where h1 is the selector (the HTML element you want to operate on)
\\color, font-size is the property
\\blue,12px is the value
\section{Types of Selectors}
\subsection{id(\#)}
specify a style for a \it single,unique HTML element\rm that uses the attribute id="something" in HTML
\begin {verbatim}
#para1{
text-align:center;
color:red;
}

In HTML
<p id="para1">Hello World!</p>
\end{verbatim}
\subsection{class(.)}
specify a style for multiple elements
\begin{verbatim}
.heading{
font-family: "ultraHUGEmegaFONT"
}
In HTML,
<h1 class="heading">Tis is the Heading </h1>

Specifying HTML elements that should only be affected by a class

p.heading{
font-family: "ultraHUGEmegaFONT"
}
In HTML, if I do the same thing
<h1 class="heading">Tis Nothing will happen ...Tis will not be the Heading </h1>
 Heading will only be applied to instances of p
<p class="heading"> Heading </p>

\end{verbatim}
\section{Inserting CSS Exterally}\footnote{You can also insert CSS Inline ,mutiple style sheet, internal style sheet }
\\Each page must link to style sheet using <link> tag.(inside the <head>)
\\Must be saved as .css extension
In the HTML file,
\begin{verbatim}
<head>
<link rel="stylesheet" type="text/css" href="mystyle.css">
</head>
\end{verbatim}
This one is the change that is actually applied to my browser    
\begin{verbatim}
background: linear-gradient(to bottom,  
    #35fcad 0%,  #73e3ff 100%); /* W3C, IE10+ */
\end{verbatim}

\end{document}
